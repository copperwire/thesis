\section{Hyperparameters}\label{sec:hyperparams}

In the previous section on regularization, we introduced the regularization strength $\lambda$ without much fanfare. However, that innocuous-seeming inclusion has quite far-reaching consequences. All of which stem from the question of how we should determine the value of the parameter $\lambda$. There is no analytical solution for the optimal value, and so we are left with other ways of estimating its value. Either by educated guesswork or through some principled schema. More sophisticated models require several of these parameters. Moreover, they will often strongly affect optimization. Collectively these parameters are known as hyperparameters.

Hyperparameters are parameters in the model that have to be empirically determined. These parameters also often impact performance and do not have a closed-form derivative in the optimization problem. These parameters have proven to be vitally important to the optimization of machine learning models. In the simple linear or logistic regression case the hyperparameters include the choice of regularization norm (ordinarily the $L_1$ or $L_2$ norms) and the regularization strength $\lambda$. The choices of all these parameters are highly non-trivial because their relationship can be strongly co- or contra-variant. Additionally, the search for these parameters is expensive because each configuration of parameters requires re-training the model.

In this section, we will discuss the general process of tuning hyperparameters in general. Subsequently,  we will introduce specific parameters that need tuning that pertain to particular architectures used in this thesis. Whichever scheme is chosen for the optimization they each follow the same basic procedure:

\begin{enumerate}
\item Choose hyperparameter configuration
\item Train model
\item Evaluate performance
\item Log performance and configuration
\end{enumerate}

\noindent When searching for hyperparameter configurations for a given model, it becomes necessary to define a scale for the variable. Together with the range, the scale defines the interval on which we search. That is, the scale defines the interval width on the range of the parameter. Usually, the scale is either linear or logarithmic, though some exceptions exist. As we introduce hyperparameters for each model, a suggested scale will also be discussed.

\subsection{Hand holding}
The most naive way of doing hyperparameter optimization is to tune the values by observing changes in performance metrics manually. This approach is very naive and rarely used in modern modeling pipelines outside the prototyping phase. For this thesis, we use a handheld approach to get a rough understanding of the ranges of values over which to apply a well-reasoned approach. 

\subsection{Grid Search}
Second on the ladder of naiveté is the exhaustive search of hyperparameter configurations. In this paradigm, one defines a multi-dimensional grid of parameters over which the model we evaluate the model. This approach has two principal pitfalls, the first is computational: If one has a set of $N$ magnitudes of the individual parameter sets $s = \{n_i\}_{i=0}^{N}$ with values of the individual parameters $\gamma_k$ and where $n_i = |\{\gamma_k\}|$ then the complexity of this search is $\mathcal{O}(\prod_{n \in s} n)$. For example we want to estimate the learning rate $\eta = \{\eta_k\}$ and the regularization strength $\lambda = \{\lambda_k\}$, then this search is a double loop as illustrated in algorithm \ref{algo:gridsearch}. In practice, the grid search is rarely used as the computational complexity scales exponentially with the number and resolution of the parameters. The nested nature of the for loops is also extremely inefficient if the hyperparameters are disconnected.  That is, neither co- or contra-variant.  

\begin{algorithm}[H]
\SetAlgoLined
\KwData{Arrays of float values $\lambda$, $\eta$}
\KwResult{log of performance for each training}
Initialization ;\\
log $\leftarrow [\hspace{0.1in}]$;\\
\For{$\lambda_k \in \lambda$}{
    \For{$\eta_k \in \eta$}{
        opt $\leftarrow$ optimizer($ \eta _k $);\\
        model\_instance $\leftarrow$ model($\lambda _k$);\\
        metrics $\leftarrow$ model\_instance.fit($\boldsymbol{X}, \boldsymbol{\hat{y}}$, opt) ;\\
        log.append((metrics, ($\lambda_k$, $\eta_k$)))
    }
}
\caption{Showing a grid search hyperparameter optimization for two hyperparameters $\eta$ and $\lambda$}\label{algo:gridsearch}
\end{algorithm}

\subsection{Random Search}
Surprisingly, one of the hyperparameter search algorithms that proved to be among the most effective is a simple random search. \citet{Bergstra2012} showed the inefficiency of doing grid search empirically and proposed the simple remedy of doing randomized configurations of hyperparameters. The central argument of the paper is elegantly presented in figure \ref{fig:randomsearch}. Where the authors observe that a grid search is both more computationally expensive and has significant shortcomings for complex modalities in the loss functions. As such, we approach the majority of hyperparameter searches in this thesis by way of random searches. The algorithm for the random search is straightforward. It requires one to draw a configuration of hyperparameters and run a fitting procedure $N$ times and log the result. In terms of performance, both grid and random search can be parallelized with linear speedups. 

\begin{figure}[ht]
\centering
\hspace{-0.5cm}
\includegraphics[height=5cm]{../figures/randomsearch.png}
\caption[Illustrating why randomsearch works]{Figure showing the inefficiency of grid search. The shaded areas on the axis represent the unknown variation of the cost as a function of that axis-parameter. Since the optimum of the loss with respect to a hyperparameter might be very narrowly peaked a grid search might miss the optimum entirety.  A random search is less likely to make the same error as the parameter selection is not regular. Figure from \citet{Bergstra2012}}\label{fig:randomsearch}
\end{figure} 
