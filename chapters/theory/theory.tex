\chapter{Theory}

The hypothesis explored in this thesis is that we can extract compressed information about physical events from the AT-TPC experiment using modern machine learning methods. With the underlying research question asking what information are we able to capture about nuclear events in the AT-TPC without needing hand-labeling of data. To achieve this we employ the DRAW algorithm (\cite{Gregor2015}). The DRAW algorithm is built of neural network components in a joint architecture comprised of a variational autoencoder wrapped in a long term short term memory cell. Each  of the components are discussed in their own sections starting with the neural network in  section \ref{sec:ANN} then followed by autoencoders in section \ref{sec:autoencoder} and finally recurrent neural networks in \ref{sec:rnn}. 

\noindent To arrive at the DRAW network we need to introduce the optimization of the log likelihood function using a binary cross-entropy cost function. In it's simplest form this optimization problem occurs in the formulation of the logistic regression mode introduced in section \ref{sec:LogReg}. As part of the derivation of the variational autoencoder cost the same optimization problem of the log likelihood will be applied. Likewise we introduce gradient descent methods and regularization, crucial components of modern machine learning, in the familiar framework of linear regression in section \ref{sec:LinReg}. 

\noindent Furthermore we hypothesize that this compressed information can be used to linearly separate events in classes. The linear separation of data serves as a precursor to using clustering algorithms to entirely remove the need of researcher intervention in labeling AT-TPC data. In the experiment at hand we hope to separate proton and carbon elastic scattering in a magnetic field. The strength of this model is that it does not rely on hand-labeled data. We demonstrate this first by applying the DRAW algorithm to simulated data showing that we can construct very good separations before turning to real data.

