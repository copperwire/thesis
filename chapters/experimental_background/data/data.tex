\section{Data}\label{sec:data}

In this thesis we will work with data from the ${}^{46}Ar(p, p')$ experiment conducted at the NSCL (national superconducting cyclotron laboratory) located on the Michigan state university campus. We principally work with two different data sources: data from AT-TPC simulation tools, and data recorded in the AT-TPC experiment proper. The experimental data was recorded from a single run of the experiment, which yields on the order of $\sim 10^4$ events.	For the simulated data we construct two datasets on the order of $\sim10^3$ and $\sim10^4$ events, respectively. In this section we give a brief overview of the data, for a more in depth consideration we refer to \cite{Mittig2015}, \cite{Suzuki2012} and  \cite{Bradt2017a}. 

In this thesis we will explore the machine learning algorithms described in chapters \ref{ch:ml} and \ref{ch:autoencoder} on three datasets from the AT-TPC, listed in table \ref{tab:datasets}. The simulated data will serve as a baseline for performance, while the filtered and full data are real records from the ${}^{46}Ar$ experiment. The former of which has some post-processing applied to remove noise, and the latter having no such processing applied. The details of the post-processing are given in later sections, first we consider the pipeline from raw data to images that the algorithms can process. 

\begin{table}[H]
\centering
\caption{Descriptions of number of events in the data used for analysis in this thesis. In principle we can simulated infinite data, but it is both quite simple and not very interesting outside a case for a proof-of-concept}\label{tab:datasets}
\begin{tabular}{lccc}
\toprule
{} & Simulated & Full & Filtered \\
\midrule
Total &  $8000$ & $51891$ & $49169$ \\
Labeled & $2400$ & $1774$ &  $1582$ \\ 
\bottomrule
\end{tabular}
\end{table}

\subsection{Data processing}

Our data processing pipeline begins with localized point-cloud data in the two dimensional detector coordinate system, with one time dimension and a charge measurement. The charge and time measurement are extracted as the peak of this signal over the event, resulting in a maximum of one measurement per pad in the sensor plane. The events range from fairly sparse $< 10^2$ records to being very populated, depending on where in the volume the reaction occurs as well as other noise-generating factors. We center the charge data to values $>1$ by adding the lowest occurring record in the run, and apply a log-transform to get values in $\R^+$. Subsequently, we scale by the maximum value in the run to get charge data in the interval $[0, 1]$. Lastly the transformed charge data is saved in a two dimensional array using the \lstinline{matplotlib} package provided in the \lstinline{Python} programing language (\cite{matplotlib}). The choice of scaling was made to accommodate a binary cross-entropy loss on a 2D projection, as it presumes the true values to be bounded as probabilities.

We will begin by considering the simulated events, followed by subsequent considerations of the full and filtered experimental data.

\subsection{Simulated \texorpdfstring{${}^{46}Ar$}{46Ar}  events}\label{sec:data_sim}

The simulated AT-TPC tracks were simulated with the \lstinline{pytpc} package developed at the NSCL (\cite{Bradt2017a}). Using the same parameters as for the $Ar^{46}(p, p)$ experiment a small set of $N=4000$ events were generated per class, as well as a larger set of $N=80000$. The events are generated as point-clouds, consisting of position data on the x-y plane, a time-stamp and an associated charge value. These point-clouds are transformed to pure x-y projections with charge intensity for the analysis in this thesis. This description is entirely analogous to the real experimental data. Using python plotting tools we transform the data to two dimensional matrices with the axes representing the x-y plane, binned in $M=128$ discrete buckets. The values in this matrix are charge values, which we log-scale and normalize to the $[0, 1]$ range. 

More formally the events are originally composed of peak-only 4-tuples of $e_i = (x_i, y_i, t_i, c_i)$. The peak-only designation indicates that we use the recored peak amplitude on each pad, the tuples then correspond to pads that recorded a signal for that event. Each event is then a set of these four-tuples: $\epsilon_j = \{e_i\}$ creating a track in three dimensional space with charge amplitude for each point. To process these events with the algorithms implemented for this thesis we chose to represent these 3D tracks as 2D images with charge represented as pixel images. For the analysis we chose to view the x-y projection of the data.

To emulate the real-data case we set a subset of the simulated data to be labeled and treat the rest as unlabeled data. We chose this partition to be $15\%$ of each class. We denote this subset and its associated labels as $\gamma_L=(\mathbf{X}_L, \mathbf{y}_L)$, the entire dataset which we will denote as $\mathbf{X}_F$. To clarify please note that $\mathbf{X}_L \subset \mathbf{X}_F$.

We display two simulated events in figure \ref{fig:sim_samples}. The top row illustrates a proton-event, and the bottom a carbon-event. 

\begin{figure}[H]
\centering
\includegraphics[width=\textwidth]{../plots/display_eventssimulated.pdf}
\caption[Displaying simulated events in 2D and 3D]{Two- and three-dimensional representations of two events from a simulated ${}^{46}Ar$ experiment. Each row is one event in two projections, where the lightness of each point indicates higher charge values.}\label{fig:sim_samples}
\end{figure}


\subsection{Full \texorpdfstring{${}^{46}Ar$}{46Ar}  events}\label{sec:data_real}

The events analyzed in this section were retrieved from the on-going AT-TPC experiment at Michigan State University. In the experiment a beam of a particular isotope is accelerated and directed into a chamber with a gas that acts as the reaction medium and target. As reactions occur between the gas and beam, ejected electrons from these drift towards the anode and the Micromegas. The Micromegas measures the impact over time from the reactions.

The measuring apparatus is very sensitive, as such there is substantial noise in the ${}^{46}Ar$ data. The noise can be attributed to structural noise from electronics cross-talk, and possible interactions with cosmic background radiation, as well as other sources of charged particles. Part of the challenge for this data then comes from understanding of the physics of the major contributing factors to this noise. 

We display two different events from the ${}^{46}Ar$ experiment in figure \ref{fig:samples}. The top row illustrates an event with a large fraction of noise, while the bottom row shows an event nearly devoid of noise. The very clear spiral structure of the bottom row indicates that this is a proton-event.

\begin{figure}[H]
\centering
\includegraphics[width=\textwidth]{../plots/display_eventsfull_.pdf}
\caption[Displaying un-filtered events in 2D and 3D]{Two- and three-dimensional representations of two events from the ${}^{46}Ar$ experiment. Each row is one event in two projections, where the lightness of each point indicates higher charge values.}\label{fig:samples}
\end{figure}

\subsection{Filtered \texorpdfstring{${}^{46}Ar$}{46Ar} events}\label{sec:filtered}

As we saw in the previous section the detector picks up significant amounts of noise. The noise can be broadly attributed to random-uncorrelated noise and structured noise. The former can be quite trivially removed with a nearest-neighbor algorithm that checks if a tuple is close to any other. To remove the correlated noise researchers at the NSCL developed an algorithm based on the Hughes' transform. This transformation is a common technique in computer vision, used to identify common geometric shapes like lines and circles. In essence the algorithm draws many lines (of whatever desired geometry) and checks whether this line intersects with other points in the data-set. These algorithms remove a large amount of the un-structured noise, and is computationally rather cheap.

We illustrate two filtered events in figure \ref{fig:samples_filtered}. These are the same events as shown in figure \ref{fig:samples}, but with the Hughes' and nearest neighbors filtering applied. 

\begin{figure}[H]
\centering
\includegraphics[width=\textwidth]{../plots/display_eventsclean_.pdf}
\caption[Displaying filtered events in 2D and 3D]{Two- and three-dimensional representations of two events from the ${}^{46}Ar$ experiment. Each row is one event in two projections, where the lightness of each point indicates higher charge values. These events have been filtered with a nearest neighbors algorithm and a Hughes' transform, described in section \ref{sec:filtered}}\label{fig:samples_filtered}
\end{figure}

