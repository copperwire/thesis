\chapter{Experimental Background}\label{ch:experimental}

The experiment which is the topic of analysis in this thesis was conducted at the facility for rare isotope beams (FRIB) located on the Michigan state university (MSU) campus. As the name implies the FRIB offers researchers the ability to study rare isotopes. These isotopes are short lived, and not normally found on earth. Applications of the studies conducted at the FRIB include furthering the understanding of nuclear structure, and nuclear astrophysics, as well as having applications in medicine and industry. 

The rarity of the nucleus under scrutiny necessitates a detector with a very high efficiency, one of which is the active target time projection chamber (AT-TPC). While in some high-energy applications the reactions of interest can be extracted from statistical properties, the low cross section of relevant interactions necessitates an individual consideration of each reaction. The goal of the analysis is then to extract as many of the events of interest, while keeping the sample as pure as possible. 

Traditional methods of extraction have been centered around Markov chain Monte Carlo (MCMC) algorithms. The fitting procedure presumes that each event is a reaction of interest, and performs an integration over the reaction vertex, initial momenta etc. Subsequently, a threshold for the fits was computed to extract the events of interest. This approach has two fundamental challenges, both related to the process of fitting against the positive class. Firstly, if the breadth of reactions is not known prior to the analysis, fitting against the positive class might yield unexpected results. Secondly, the fitting presumes complete tracks in the generated point-cloud of the event. In the event of broken tracks the fitting will not converge satisfactorily. Additionally, the computational cost of fitting each track with a MCMC algorithm is prohibitively large. 

The foray into machine learning is then an attempt to address these challenges. This started with the work by \cite{Kuchera2019} in which the authors successfully train high performing classifiers for the ${}^{46}Ar$ experiment performed with the AT-TPC detector. In this thesis we elaborate on these results by introducing unsupervised techniques for the separation of reactions in the same experiment.