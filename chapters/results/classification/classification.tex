\chapter{Classification results}\label{chap:classification}

To prime our discussion on clustering algorithms on AT-TPC data we first investigate the easier problem of semi-supervised classification. The goal of this analysis is to determine whether we are able to construct latent spaces that separate the known event-types from the ${}^{46}Ar$ experiment. We investigate the latent space of a pre-trained model and two different autoencoder structures. Each of these models are evaluated on three datasets: simulated, clean, and real AT-TPC events. The evaluation is performed by training a logistic regression classifier on the latent samples, and performance is measured by the $f1$ score. 

The training procedure for classification using a semi-supervised regime as the one we'll apply necessitates the same strict separation of labeled data for the classification step as when considering ordinary classification tasks. Details on the modeling pipeline can be found in section \ref{sec:architecture}. All models excepting the baseline pre-trained VGG model were tuned with the \lstinline{RandomSearch} architecture, which searches in a semi structured way over all the parameters given in table \ref{tab:convae_hyperparams}. As a benchmark we start by measuring the performance using just the pre-trained VGG16 representation of the labeled data of each dataset. The two proposed representation algorithms are then presented with results for each dataset for comparison. 
