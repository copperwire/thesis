% !TEX spellckeck=en_GB

\chapter{Classification results}\label{chap:classification}

To prime our discussion on clustering algorithms for AT-TPC data, we first consider the easier problem of semi-supervised classification. The goal of this analysis is to determine whether we can construct latent spaces that separate the known event-types from the ${}^{46}$Ar experiment. We investigate the latent space of a pre-trained model and two different autoencoder structures. Subsequently, we evaluate each of these models on three datasets: simulated, filtered, and full AT-TPC events. The evaluation is performed by training a logistic regression classifier on the latent samples, and we measure performance by the $f1$ score. 

The training procedure for classification using a semi-supervised regime necessitates the same strict separation of labelled data for the classification step as when considering ordinary classification tasks. Details on the modelling pipeline are found in section \ref{ch:architectures}. We tuned all models excepting the baseline pre-trained VGG model with the \lstinline{RandomSearch} architecture, which searches in a semi structured way over all the parameters given in table \ref{tab:convae_hyperparams}. As a benchmark, we start by measuring the performance using just the pre-trained VGG16 representation of the labelled data of each dataset. The two proposed representation algorithms are then presented with results for each dataset for comparison. 
