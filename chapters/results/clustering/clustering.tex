\chapter{Clustering of AT-TPC events}\label{chap:clustering}

The principal challenge in the AT-TPC experiment that we're trying to solve is the the reliance on labeled samples. While the ${}^{46}Ar$ experiment has conveniently dissimilar reaction products, which facilitates supervised learning, this is not so for other experiments. However, the ${}^{46}Ar$ experiment provides a convenient example where we can explore unsupervised techniques. In this chapter we explore the application of clustering techniques to events represented in latent spaces. 

We begin by exploring a naive K-means approach on the latent space of a pre-trained network. Subsequently we investigate other clustering methods and two autoencoder based clustering algorithms as outlined in section \ref{sec:deep_clustering}.

This chapter builds on the previous results from semi-supervised classification. We observe that we are able to construct high quality latent spaces, which is enables the investigation of clustering analysis. 

The approach to the clustering of events is different to the semi-supervised approach in a couple of meaningful ways. Firstly it's a harder task, as we'll demonstrate, which necessitates a more exploratory approach to the problem. Secondly as a a consequence of the challenge the focus will be a bit different than for the semi-supervised approach. We will still utilize the same architectures and models starting with a search over the parameter space over which we measure the performance using the ARS (adjusted rand score) and accuracy defined in section \ref{sec:unsupervised_perf} and \ref{sec:supervised_perf}, respectively.

As with the chapter on the semi-supervised results we start with considering the VGG16 pre-trained model as a benchmark, but owing to the challenge presented by clustering analysis we will compare and contrast with results achieved on MNIST. Wile clustering the handwritten digits in the MNIST dataset is a relatively simple task, it can help identify what challenges we face when clustering AT-TPC events. 

We also note that this chapter does not perform rigorous clustering analysis. The purpose here is exploration, and so the focus is largely on discovering possible avenues for further research rather than identifying exactly what promise those avenues hold. 