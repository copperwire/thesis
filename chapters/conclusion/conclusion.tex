\chapter{Conclusions and Future Work}


In the present work, we explored the segmentation of active target time-projection chamber (AT-TPC) events in neural network latent spaces. This exploration is necessary because traditional methods are both computationally prohibitively expensive and can not be applied to events with broken tracks. Specifically, the goal was to implement autoencoder based models for semi-supervised classification and clustering. We compared these models with a classic pre-trained model from the image analysis community.

Two tasks were proposed to contribute to the exploration of  AT-TPC events: a semi-supervised objective which describes the necessary volume of labelled data, and a clustering objective which measures the quality of segmentation without labelled data. 

To solve the semi-supervised problem, we implemented two autoencoder-based algorithms: a convolutional autoencoder model with the capacity for two different latent space regularisation, and the sequential deep recurrent attentive writer (DRAW) model. We trained these models on three different sets of AT-TPC data and found the following:

\begin{itemize}
\item A convolutional autoencoder can linearly segment its latent space by event type when trained on the ${}^{46}Ar$ data.
\item Good class separability can be achieved both with and without latent regularisation.  When regularised, we found better segmentation in models trained with a Gaussian mixture maximum mean discrepancy loss, than those trained with a variational autoencoder loss. 
\item The recurrent DRAW  model does not offer meaningful improvements to the convolutional autoencoder performance on the semi-supervised task. 
\end{itemize}

Moreover, neither the DRAW algorithm nor the convolutional autoencoder outperformed the pre-trained VGG16 as a function of the available labelled data. This discrepancy indicates that while the reconstruction objective encourages class separability in the latent space, it does not do so to the degree that a classification objective does even when the classification objective is over a different dataset. 

Avenues of academic interest for further research include the construction of new representations for the duelling decoder objective, as well as models that combine autoencoders with generative adversarial networks.  Lastly, we analysed two-dimensional projections of AT-TPC events in this work, expanding to include the full three-dimensional cold contribute additional insight.

To address the clustering task, we implemented two algorithms: the deep convolutional embedded clustering (DCEC) algorithm and the mixture of autoencoders algorithm. As in the semi-supervised objective, we compared these algorithms with the performance of a pre-trained VGG16 network. We showed that the pre-trained network latent space could be combined with a simple k-means algorithm for clustering of AT-TPC events. With the VGG16+K-means algorithm, we achieved convincing results on simulated data, as well as promising segmentation of the full and filtered data. Especially notable was the consistent purity of the proton event cluster. With the autoencoder based MIXAE  algorithm, we found that an increase in the performance on the clustering task from the VGG16+K-means approach.  In conclusions, we found the following for the clustering task:

\begin{itemize}
\item Using a K-means algorithm on the VGG16 latent space, we demonstrate strong clustering of simulated data. Additionally, this approach consistently finds high purity proton clusters for both filtered and full ${}^{46}Ar$ data.
\item Building on the insights from the semi-supervised task, and the failure of the DCEC algorithm,  we successfully clustered AT-TPC data with the MIXAE algorithm. We demonstrate that this approach can increase performance on the clustering task compared to the VGG16+K-means algorithm.  However, the MIXAE performance is dependent on the loss-weights which we selected based on the performance on the clustering task. We also found challenges with the MIXAE model as it has significant stability problems. 
\end{itemize}

The contribution of the present work is then to both demonstrate the applicability of pre-trained models in the unsupervised clustering of AT-TPC data. Moreover, we have shown that MIXAE model can improve upon this performance. Further research is needed to understand the variability in autoencoder based clustering performance. Additionally, deep clustering is an active field of research and novel methods might provide additional insight. 

In summary, we have found promising avenues for research applying both supervised and unsupervised techniques applied to AT-TPC data. With the latter having major implications for experiments in which researchers are unable to separate event types. However, this research is still at an early stage and for current experiments we recommend the application of pre-trained models for both supervised and unsupervised tasks.