\chapter{Discussion}
In this chapter, we review the results presented in the previous chapter. The primary aim of this chapter is to frame our findings by the goals for our analysis, as presented in sections \ref{sec:clf_procedure} and \ref{sec:cster_procedure}. We divide the discussion into topics of task. First, we will consider the classification performance of our two implemented algorithms on the three different datasets: simulated, filtered and full AT-TPC data. Second, we discuss the unsupervised clustering performance on the same datasets. 

To prime or discussion of the semi-supervised classification results, we briefly restate the goals for the analysis: The core task we aim to accomplish is to describe the model performance as a function of the available labelled data. Furthermore, we wish to characterize the latent spaces produced by the different algorithms qualitatively. The performance is contextualized by the work of \citet{Kuchera2019}, who introduced the application of pre-trained models to AT-TPC data. Lastly, the semi-supervised performance serves as a proof-of-concept for the construction of high-quality latent spaces in an AT-TPC experiment. This proof-of-concept spurred the implementation of autoencoder based algorithms for clustering of AT-TPC data.